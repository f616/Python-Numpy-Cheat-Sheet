%-----------------------------------------------------
\begin{alerttextbox}{Introductory Note}
This document is an adaption of the original datacamp.org cheat sheet.\\
\begin{itemize}
    \item {https://www.datacamp.com/resources/cheat-sheets/numpy-cheat-sheet-data-analysis-in-python}
    \item {https://github.com/f616/Python-Numpy-Cheat-Sheet}
\end{itemize}

\end{alerttextbox}


%-----------------------------------------------------
\section{Numpy}

\begin{myblock}{}
The NumPy library is the core library for scientific computing in Python.\\
It provides a high-performance multidimensional array object, and tools for working with these arrays.\\

\textbf{Use the following import convention:}
\begin{codebox}{python}{}
import numpy as np
\end{codebox}

\begin{myblock}{NumPy Arrays}
\mygraphics{img/numpy-arrays.png}
\end{myblock}

\end{myblock}

%--------------------------------------------------------------
\section{Creating Arrays}

\begin{codebox}{python}{}
a = np.array([1,2,3])
b = np.array([(1.5,2,3), (4,5,6)], dtype = float)
c = np.array([[(1.5,2,3), (4,5,6)],[(3,2,1), (4,5,6)]], dtype = float)
\end{codebox}

\begin{codebox}{python}{Initial Placeholders}
np.zeros((3,4))  #Create an array of zeros
np.ones((2,3,4),dtype=np.int16)  #Create an array of ones
d = np.arange(10,25,5)  #Create an array of evenly spaced values (step value)
np.linspace(0,2,9)  #Create an array of evenly spaced values (number of samples)
e = np.full((2,2),7)  #Create a constant array
f = np.eye(2)  #Create a 2X2 identity matrix
np.random.random((2,2))  #Create an array with random values
np.empty((3,2))  #Create an empty array
\end{codebox}


%--------------------------------------------------------------
\section{I/O}

\begin{codebox}{python}{Saving \& Loading On Disk}
np.save('my_array', a)
np.savez('my_array.npz', a, b)
np.load('my_array.npy')
\end{codebox}

\begin{codebox}{python}{Saving \& Loading Text Files}
np.loadtxt('myfile.txt')
np.genfromtxt('my_file.csv', delimiter=',')
np.savetxt('myarray.txt', a, delimiter=' ')
\end{codebox}


%--------------------------------------------------------------
\section{Asking For Help}

\begin{codebox}{python}{}
np.info(np.ndarray.dtype)
\end{codebox}


%--------------------------------------------------------------
\section{Inspecting Your Array}

\begin{codebox}{python}{}
a.shape  #Array dimensions
len(a)  #Length of array
b.ndim  #Number of array dimensions
e.size  #Number of array elements
b.dtype  #Data type of array elements
b.dtype.name  #Name of data type
b.astype(int)  #Convert an array to a different type
\end{codebox}


%--------------------------------------------------------------
\section{Data Types}

\begin{codebox}{python}{}
np.int64  #Signed 64-bit integer types
np.float32  #Standard double-precision floating point
np.complex  #Complex numbers represented by 128 floats
np.bool  #Boolean type storing TRUE and FALSE values
np.object  #Python object type
np.string_  #Fixed-length string type
np.unicode_  #Fixed-length unicode type
\end{codebox}


%--------------------------------------------------------------
\section{Array Mathematics}

\begin{codebox}{python}{Arithmetic Operations}
g = a - b  #Subtraction
>>> array([[-0.5, 0. , 0. ], [-3. , -3. , -3. ]])
np.subtract(a,b)  #Subtraction

b + a  #Addition
>>> array([[ 2.5, 4. , 6. ], [ 5. , 7. , 9. ]])
np.add(b,a)  #Addition

a / b  #Division
>>> array([[ 0.66666667, 1. , 1. ], [ 0.25 , 0.4 , 0.5 ]])
np.divide(a,b)  #Division

a * b  #Multiplication
>>> array([[ 1.5, 4. , 9. ], [ 4. , 10. , 18. ]])
np.multiply(a,b)  #Multiplication

np.exp(b)  #Exponentiation
np.sqrt(b)  #Square root
np.sin(a)  #Print sines of an array
np.cos(b)  #Element-wise cosine
np.log(a)  #Element-wise natural logarithm
e.dot(f)  #Dot product
>>> array([[ 7., 7.], [ 7., 7.]])
\end{codebox}

\begin{codebox}{python}{Comparison}
a == b  #Element-wise comparison
>>> array([[False, True, True], [False, False, False]], dtype=bool)

a < 2  #Element-wise comparison
>>> array([True, False, False], dtype=bool)

np.array_equal(a, b)  #Array-wise comparison
\end{codebox}

\begin{codebox}{python}{Aggregate Functions}
a.sum()  #Array-wise sum
a.min()  #Array-wise minimum value
b.max(axis=0)  #Maximum value of an array row
b.cumsum(axis=1)  #Cumulative sum of the elements
a.mean()  #Mean
b.median()  #Median
a.corrcoef()  #Correlation coefficient
np.std(b)  #Standard deviation
\end{codebox}


%--------------------------------------------------------------
\section{Copying Arrays}

\begin{codebox}{python}{}
h = a.view()  #Create a view of the array with the same data
np.copy(a)  #Create a copy of the array
h = a.copy()  #Create a deep copy of the array
\end{codebox}


%--------------------------------------------------------------
\section{Sorting Arrays}

\begin{codebox}{python}{}
a.sort()  #Sort an array
c.sort(axis=0)  #Sort the elements of an array's axis
\end{codebox}


%--------------------------------------------------------------
\section{Subsetting, Slicing, Indexing}

\begin{myblock}{Subsetting}
\begin{codebox}{python}{}
a[2]  #Select the element at the 2nd index
>>> 3
\end{codebox}
\begin{tabular}{ |c|c|c| } 
 \hhline{---}
 1 & 2 & \cellcolor[HTML]{FFFFFF}3\\
 \hhline{---}
\end{tabular}\\

\begin{codebox}{python}{}
b[1,2]  #Select the element at row 1 column 2 (equivalent to b[1][2])
>>> 6.0
\end{codebox}
\begin{tabular}{ |c|c|c| } 
 \hhline{---}
 1.5 & 2 & 3\\
 \hhline{---}
   4 & 5 & \cellcolor[HTML]{FFFFFF}6\\
 \hhline{---}
\end{tabular}
\end{myblock}

\begin{myblock}{Slicing}
\begin{codebox}{python}{}
a[0:2]  #Select items at index 0 and 1
>>> array([1, 2])
\end{codebox}
\begin{tabular}{ |c|c|c| } 
 \hhline{---}
 \cellcolor[HTML]{FFFFFF}1 & \cellcolor[HTML]{FFFFFF}2 & 3\\
 \hhline{---}
\end{tabular}\\

\begin{codebox}{python}{}
b[0:2,1]  #Select items at rows 0 and 1 in column 1
>>> array([ 2., 5.])
\end{codebox}
\begin{tabular}{ |c|c|c| } 
 \hhline{---}
 1.5 & \cellcolor[HTML]{FFFFFF}2 & 3\\
 \hhline{---}
   4 & \cellcolor[HTML]{FFFFFF}5 & 6\\
 \hhline{---}
\end{tabular}\\

\begin{codebox}{python}{}
b[:1]  #Select all items at row 0 (equivalent to b[0:1, :])
>>> array([[1.5, 2., 3.]])
\end{codebox}
\begin{tabular}{ |c|c|c| } 
 \hhline{---}
 \cellcolor[HTML]{FFFFFF}1.5 & \cellcolor[HTML]{FFFFFF}2 & \cellcolor[HTML]{FFFFFF}3\\
 \hhline{---}
   4 & 5 & 6\\
 \hhline{---}
\end{tabular}\\

\begin{codebox}{python}{}
c[1,...]  #Same as [1,:,:]
>>> array([[[ 3., 2., 1.],
            [ 4., 5., 6.]]])
\end{codebox}

\begin{codebox}{python}{}
a[ : :-1]  #Reversed array a array([3, 2, 1])
\end{codebox}
\end{myblock}

\begin{myblock}{Boolean Indexing}
\begin{codebox}{python}{}
a[a<2]  #Select elements from a less than 2
>>> array([1])
\end{codebox}
\begin{tabular}{ |c|c|c| } 
 \hhline{---}
 \cellcolor[HTML]{FFFFFF}1 & 2 & 3\\
 \hhline{---}
\end{tabular}
\end{myblock}

\begin{myblock}{Fancy Indexing}
\begin{codebox}{python}{}
b[[1, 0, 1, 0],[0, 1, 2, 0]]  #Select elements (1,0),(0,1),(1,2) and (0,0)
>>> array([ 4. , 2. , 6. , 1.5])
\end{codebox}
\begin{codebox}{python}{}
b[[1, 0, 1, 0]][:,[0,1,2,0]]  #Select a subset of the matrix’s rows and columns
>>> array([[ 4. ,5. , 6. , 4. ],
           [ 1.5, 2. , 3. , 1.5],
           [ 4. , 5. , 6. , 4. ],
           [ 1.5, 2. , 3. , 1.5]])
\end{codebox}
\end{myblock}


%--------------------------------------------------------------
\section{Array Manipulation}

\begin{codebox}{python}{Transposing Array}
i = np.transpose(b)  #Permute array dimensions
i.T  #Permute array dimensions
\end{codebox}

\begin{codebox}{python}{Changing Array Shape}
b.ravel()  #Flatten the array
g.reshape(3,-2)  #Reshape, but don’t change data
\end{codebox}

\begin{codebox}{python}{Adding/Removing Elements}
h.resize((2,6))  #Return a new array with shape (2,6)
np.append(h,g)  #Append items to an array
np.insert(a, 1, 5)  #Insert items in an array
np.delete(a,[1])  #Delete items from an array
\end{codebox}

\begin{codebox}{python}{Combining Arrays}
np.concatenate((a,d),axis=0)  #Concatenate arrays
>>> array([ 1, 2, 3, 10, 15, 20])

np.vstack((a,b))  #Stack arrays vertically (row-wise)
>>> array([[ 1. , 2. , 3. ],
           [ 1.5, 2. , 3. ],
           [ 4. , 5. , 6. ]])

np.r_[e,f]  #Stack arrays vertically (row-wise)

np.hstack((e,f))  #Stack arrays horizontally (column-wise)
>>> array([[ 7., 7., 1., 0.],
           [ 7., 7., 0., 1.]])

np.column_stack((a,d))  #Create stacked column-wise arrays
>>> array([[ 1, 10],
           [ 2, 15],
           [ 3, 20]])

np.c_[a,d]  #Create stacked column-wise arrays
\end{codebox}

\begin{codebox}{python}{Splitting Arrays}
np.hsplit(a,3)  #Split the array horizontally at the 3rd index
>>> [array([1]),array([2]),array([3])]

np.vsplit(c,2)  #Split the array vertically at the 2nd index
>>> [array([[[ 1.5, 2. , 1. ], [ 4. , 5. , 6. ]]]),
     array([[[ 3., 2., 3.], [ 4., 5., 6.]]])]
\end{codebox}
